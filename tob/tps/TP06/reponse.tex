% Options for packages loaded elsewhere
\PassOptionsToPackage{unicode}{hyperref}
\PassOptionsToPackage{hyphens}{url}
%
\documentclass[
]{article}
\usepackage{amsmath,amssymb}
\usepackage{lmodern}
\usepackage{ifxetex,ifluatex}
\ifnum 0\ifxetex 1\fi\ifluatex 1\fi=0 % if pdftex
  \usepackage[T1]{fontenc}
  \usepackage[utf8]{inputenc}
  \usepackage{textcomp} % provide euro and other symbols
\else % if luatex or xetex
  \usepackage{unicode-math}
  \defaultfontfeatures{Scale=MatchLowercase}
  \defaultfontfeatures[\rmfamily]{Ligatures=TeX,Scale=1}
\fi
% Use upquote if available, for straight quotes in verbatim environments
\IfFileExists{upquote.sty}{\usepackage{upquote}}{}
\IfFileExists{microtype.sty}{% use microtype if available
  \usepackage[]{microtype}
  \UseMicrotypeSet[protrusion]{basicmath} % disable protrusion for tt fonts
}{}
\makeatletter
\@ifundefined{KOMAClassName}{% if non-KOMA class
  \IfFileExists{parskip.sty}{%
    \usepackage{parskip}
  }{% else
    \setlength{\parindent}{0pt}
    \setlength{\parskip}{6pt plus 2pt minus 1pt}}
}{% if KOMA class
  \KOMAoptions{parskip=half}}
\makeatother
\usepackage{xcolor}
\IfFileExists{xurl.sty}{\usepackage{xurl}}{} % add URL line breaks if available
\IfFileExists{bookmark.sty}{\usepackage{bookmark}}{\usepackage{hyperref}}
\hypersetup{
  hidelinks,
  pdfcreator={LaTeX via pandoc}}
\urlstyle{same} % disable monospaced font for URLs
\usepackage{listings}
\newcommand{\passthrough}[1]{#1}
\lstset{defaultdialect=[5.3]Lua}
\lstset{defaultdialect=[x86masm]Assembler}
\setlength{\emergencystretch}{3em} % prevent overfull lines
\providecommand{\tightlist}{%
  \setlength{\itemsep}{0pt}\setlength{\parskip}{0pt}}
\setcounter{secnumdepth}{-\maxdimen} % remove section numbering
\ifluatex
  \usepackage{selnolig}  % disable illegal ligatures
\fi

\author{}
\date{}

\begin{document}

\begin{center}\rule{0.5\linewidth}{0.5pt}\end{center}

\hypertarget{pandoc--s-reponse.md--o-reponse.pdf}{%
\subparagraph{pandoc -s reponse.md -o
reponse.pdf}\label{pandoc--s-reponse.md--o-reponse.pdf}}

lang: fr-FR title: TP 06 Tob subtitle: Héritage comme spécialisation
author: Julien Blanchon date: 08-02-2020 base: . \%Base directory for
import file abstract: blabla keywords: {[}tob, java{]} output:
pdf\_document: path: reponse.pdf \# include-before: \textbar{} \# See
these lines \# Come \textbf{before} the \href{/dev/null}{toc} toc-title:
Table des matières toc: true toc\_depth: 2 \#include-after: \textbar{}
\# See these lines \# Come \textbf{after} the \href{/dev/null}{article}
fontsize: 12pt \#10, 11 ou 12pt seulement \# Voir
https://tug.org/FontCatalogue/ et https://fonts.google.com/ pour les
Fonts \# mainfont: ``Merriweather'' \# sansfont: ``Raleway'' \#
monofont: ``IBM Plex Mono'' \# mathfont: documentclass: article \#Voir
https://en.wikibooks.org/wiki/LaTeX/Document\_Structure\#Document\_classes
pour les class et classoption classoption: {[}notitlepage, onecolumn,
openany{]} geometry: {[}a4paper, bindingoffset=0mm, inner=30mm,
outer=30mm, top=30mm, bottom=30mm{]} \# Voir
https://ctan.org/pkg/geometry pour les options geometry \# papersize: a4
header-includes: \#\#\# Format encodage -

\usepackage[utf8]{inputenc}

\begin{itemize}
\item
  \usepackage[T1]{fontenc}

  \hypertarget{format-code-ada}{%
  \subsubsection{Format Code Ada}\label{format-code-ada}}

  \begin{itemize}
  \item
    \usepackage{listings}

    \hypertarget{package-pour-les-blocks-code-httpsen.wikibooks.orgwikilatexsource_code_listings}{%
    \section{Package pour les blocks code
    (https://en.wikibooks.org/wiki/LaTeX/Source\_Code\_Listings)}\label{package-pour-les-blocks-code-httpsen.wikibooks.orgwikilatexsource_code_listings}}
  \item
    \usepackage{tcolorbox}
  \item
    \usepackage{fullpage}
  \item
    \usepackage{color}
  \item
    \definecolor{dkgreen}{rgb}{0,0.6,0}
  \item
    \definecolor{gray}{rgb}{0.5,0.5,0.5}
  \item
    \definecolor{mauve}{rgb}{0.58,0,0.82}

    \% Contents of listings-setup.tex
  \item
    \usepackage{xcolor}
  \item
    \lstset{
      basicstyle=\ttfamily,
      numbers=left,
      keywordstyle=\color[rgb]{0.13,0.29,0.53}\bfseries,
      stringstyle=\color[rgb]{0.31,0.60,0.02},
      commentstyle=\color[rgb]{0.56,0.35,0.01}\itshape,
      numberstyle=\footnotesize,
      stepnumber=1,
      numbersep=5pt,
      backgroundcolor=\color[RGB]{248,248,248},
      showspaces=false,
      showstringspaces=false,
      showtabs=false,
      tabsize=2,
      captionpos=b,
      breaklines=true,
      breakatwhitespace=true,
      breakautoindent=true,
      escapeinside={\%*}{*)},
      linewidth=\textwidth,
      basewidth=0.5em,
    }
  \item
    \lstset{
      inputencoding = utf8,
      extendedchars = true,
      literate      =      
        {á}{{\'a}}1  {é}{{\'e}}1  {í}{{\'i}}1 {ó}{{\'o}}1  {ú}{{\'u}}1
        {Á}{{\'A}}1  {É}{{\'E}}1  {Í}{{\'I}}1 {Ó}{{\'O}}1  {Ú}{{\'U}}1
        {à}{{\`a}}1  {è}{{\`e}}1  {ì}{{\`i}}1 {ò}{{\`o}}1  {ù}{{\`u}}1
          {À}{{\`A}}1  {È}{{\'E}}1  {Ì}{{\`I}}1 {Ò}{{\`O}}1  {Ù}{{\`U}}1
        {ä}{{\"a}}1  {ë}{{\"e}}1  {ï}{{\"i}}1 {ö}{{\"o}}1  {ü}{{\"u}}1
        {Ä}{{\"A}}1  {Ë}{{\"E}}1  {Ï}{{\"I}}1 {Ö}{{\"O}}1  {Ü}{{\"U}}1
        {â}{{\^a}}1  {ê}{{\^e}}1  {î}{{\^i}}1 {ô}{{\^o}}1  {û}{{\^u}}1
        {Â}{{\^A}}1  {Ê}{{\^E}}1  {Î}{{\^I}}1 {Ô}{{\^O}}1  {Û}{{\^U}}1
        {ø}{{\o}}1
      }
  \end{itemize}
\end{itemize}

\begin{center}\rule{0.5\linewidth}{0.5pt}\end{center}

\clearpage

\hypertarget{exercice-1-formaliser-le-schuxe9ma}{%
\section{Exercice 1 : Formaliser le
schéma}\label{exercice-1-formaliser-le-schuxe9ma}}

Dans la question 3.2 du TP 5, nous avons défini le schéma comme
plusieurs objets (points, points nommés et segments) qui sont référencés
par des variables différentes. Il serait plus logique et pratique
d'avoir une seule variable qui représente le schéma (on l'appellera
naturellement schema). Comme un schéma est constitué d'un nombre
variable d'éléments, on peut le représenter par un tableau. Si nous
appelons X le type des éléments de ce tableau, nous pouvons alors écrire
le code du listing 1

\begin{enumerate}
\def\labelenumi{\arabic{enumi}.}
\tightlist
\item
  Indiquer à quelles conditions sur \passthrough{\lstinline!X!} les
  lignes suivantes compilent.
\end{enumerate}

\begin{lstlisting}[language=Java]
schema[nb++] = s12;
schema[nb++] = barycentre;
schema[i].afficher();
schema[i].translater(4, -3);
\end{lstlisting}

\begin{lstlisting}[frame=single, language=Java]
class MaClasse{
  this.function{}
}
\end{lstlisting}

Il faut que \passthrough{\lstinline!X!} soit un tableau dont les
éléments ne sont pas privée pour pouvoir Point.afficher \ldots{}

\begin{enumerate}
\def\labelenumi{\arabic{enumi}.}
\setcounter{enumi}{1}
\tightlist
\item
  Quel code sera exécuté pour x.afficher() et x.translater(4, -3) ?
\end{enumerate}

\passthrough{\lstinline!Point.afficher()!} et
\passthrough{\lstinline!Point.translater(double, double)!}

\begin{enumerate}
\def\labelenumi{\arabic{enumi}.}
\setcounter{enumi}{2}
\tightlist
\item
  Indiquer les autres éléments à définir sur X ? Justifier la réponse.
\end{enumerate}

ø

\begin{enumerate}
\def\labelenumi{\arabic{enumi}.}
\setcounter{enumi}{3}
\tightlist
\item
  Donner un nom plus significatif à X.
\end{enumerate}

\passthrough{\lstinline!SchemaTab!} serait plus significatif pour X.

\hypertarget{exercice-2-uxe9crire-la-classe-x-et-adapter-lapplication}{%
\section{Exercice 2 : Écrire la classe X et adapter
l'application}\label{exercice-2-uxe9crire-la-classe-x-et-adapter-lapplication}}

\begin{enumerate}
\def\labelenumi{\arabic{enumi}.}
\tightlist
\item
  Est-ce que l'on sait écrire le code des méthodes afficher ou
  translater de X ?
\end{enumerate}

Oui si tout les éléments de X possède les méthodes afficher et
translater. Ce qui est la cas car X peut contenir soit un Point, soit un
PointNomme, soit un Segment. Il serait judicieux de faire un interface:
ObjectGeometrique.

\begin{enumerate}
\def\labelenumi{\arabic{enumi}.}
\setcounter{enumi}{1}
\tightlist
\item
  Peut-on créer des instances de X ?
\end{enumerate}

Oui ?

\begin{enumerate}
\def\labelenumi{\arabic{enumi}.}
\setcounter{enumi}{2}
\tightlist
\item
  Quels constructeurs définir sur X ?
\end{enumerate}

?

\begin{enumerate}
\def\labelenumi{\arabic{enumi}.}
\setcounter{enumi}{3}
\tightlist
\item
  Quand ces constructeurs seront-ils appelés ?
\end{enumerate}

Lors de la création de X ?

\begin{enumerate}
\def\labelenumi{\arabic{enumi}.}
\setcounter{enumi}{4}
\item
  Écrire le code de la classe X.
\item
  Lister et effectuer les modifications à apporter aux autres classes de
  l'application.
\end{enumerate}

\hypertarget{exercice-3-construire-le-schuxe9ma-en-utilisant-les-listes}{%
\section{Exercice 3 : Construire le schéma en utilisant les
listes}\label{exercice-3-construire-le-schuxe9ma-en-utilisant-les-listes}}

Au lieu d'utiliser un tableau comme dans l'exercice 2, on veut utiliser
l'interface List et sa réalisation ArrayList du paquetage java.util (en
particulier la méthode add et la structure de contrôle foreach).

\begin{enumerate}
\def\labelenumi{\arabic{enumi}.}
\item
  Indiquer les avantages et inconvénients des listes par rapport aux
  tableaux.
\item
  Construire le schéma en utilisant une liste.
\end{enumerate}

\hypertarget{exercice-4-duxe9finir-un-groupe}{%
\section{Exercice 4 : Définir un
groupe}\label{exercice-4-duxe9finir-un-groupe}}

Dans un éditeur de schémas mathématiques, il serait pratique de pouvoir
grouper plusieurs X pour les manipuler comme un seul et leur appliquer à
tous, en une seule fois, la même opération (translater, afficher, etc.).

\begin{enumerate}
\def\labelenumi{\arabic{enumi}.}
\item
  Sachant que la classe X est abstraite, la classe Groupe est-elle
  abstraite ou concrète ?
\item
  Écrire la classe Groupe et l'utiliser (ExempleSchemaGroupe).
\item
  On souhaite pouvoir mettre un groupe dans un groupe. Par exemple, on
  souhaite grouper les trois segments, puis ce groupe et le barycentre.
  Comment faire ?
\end{enumerate}

\end{document}
